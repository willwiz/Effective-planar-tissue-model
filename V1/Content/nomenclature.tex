%%%%%%%%%%%%%%%%%%%%%%%%%%%%%%%%%%%%%%%%%%%%%%%%%%%%%%%%%%%%%
%%  nomenclature											%
%%%%%%%%%%%%%%%%%%%%%%%%%%%%%%%%%%%%%%%%%%%%%%%%%%%%%%%%%%%%%

%-----------------------------------------------------------
%	Symbols
%-----------------------------------------------------------
\nomenclature[S,01]{$s$}{Scalar variables}
\nomenclature[S,02]{$\mathbf{v}$}{Vector variables, bold lower case}
\nomenclature[S,03]{$\mathbf{M}$}{Matrix variables, bold upper case}
\nomenclature[S,04]{$\hat{\Psi},\hat{S},\hat{f}$}{Data}
\nomenclature[S,05]{$\mathbf{I}$}{Identity tensor}
\nomenclature[S,06]{$J$}{The Jacobian for volume change due to deformation}
\nomenclature[S,07]{$\mathbf{F}$}{The deformation gradient tensor}
\nomenclature[S,08]{$\mathbf{f}$}{The upper triangular decomposition of the deformation gradient tensor}
\nomenclature[S,09]{$\mathbf{C}$}{Right Cauchy-Green strain tensor}
\nomenclature[S,10]{$\mathbf{B}$}{Left Cauchy-Green strain tensor}
\nomenclature[S,11]{$\mathbf{E}$}{Green Lagrange strain}
\nomenclature[S,12]{$\mathbf{U}$}{Right stretch tensor}
\nomenclature[S,13]{$\mathbf{T}$}{The Cauchy stress tensor}
\nomenclature[S,14]{$\mathbf{S}$}{Second Piola Kirchhoff tensor}
\nomenclature[S,15]{$I_1$}{First invariant of the right Cauchy strain tensor}
\nomenclature[S,16]{$I_2$}{Second invariant of the right Cauchy strain tensor}
\nomenclature[S,17]{$I_3$}{Third invariant of the right Cauchy strain tensor}
\nomenclature[S,18]{$I_4$}{Fourth pseudo-invariant of the right Cauchy strain tensor describing the stretch along an axis}
\nomenclature[S,19]{$I_8$}{Eighth pseudo invariant, describing the relative stretch along two axes}
\nomenclature[S,20]{$I_8^{ext}$}{The extensional component of $I_8$ }
\nomenclature[S,21]{$\mathbf{m}_0$}{The material axis in the reference configuration}
\nomenclature[S,22]{$\mathbf{n}_0$}{The perpendicular axis to the material axis in the reference configuration}
\nomenclature[S,23]{$\mathbf{m}_t$}{The material axis in the deformed configuration}
\nomenclature[S,24]{$\mathbf{n}_t$}{The perpendicular axis to the material axis in the deformed configuration}
\nomenclature[S,25]{$\lambda_m$}{The stretch along the material axis}
\nomenclature[S,26]{$\lambda_n$}{The stretch perpendicular to the material axis}
\nomenclature[S,27]{$\phi$}{The shear angle between $\mathbf{m}_0$ and $\mathbf{n}_0$}
\nomenclature[S,28]{$E_m, E_n, E_\phi$}{The Green Lagrange strain along the respective axes and to shearing}
\nomenclature[S,29]{$S_m, S_n, S_\phi$}{The 2nd Piola Kirchhoff stress along the respective axes and to shearing, which are also response functions, gradients of the strain energy}
\nomenclature[S,30]{$\gamma_1,\gamma_2,\gamma_3$}{The Hencky strains}
\nomenclature[S,31]{$\Psi$}{The strain energy}
\nomenclature[S,32]{$\Psi_\mathrm{col}, \Psi_\mathrm{int}, \Psi_\mathrm{mat}$}{Strain energy of the collagen fiber, ensemble-ensemble interactions, and matrix components respectively}
\nomenclature[S,33]{$\eta_C$, $\eta_M$, $\eta_I$}{The modulus of collagen, matrix and fiber-fiber interactions}
\nomenclature[S,34]{$\phi_\mathrm{col}$, $\phi_\mathrm{mat}$, $\phi_\mathrm{int}$}{The mass fractions of collagen, matrix and fiber-fiber interactions}
\nomenclature[S,35]{$D$}{Collagen fiber recruitment distribution function}
\nomenclature[S,36]{$\Gamma$}{Collagen fiber orientation distribution function}
\nomenclature[S,37]{$\lambda$}{Stretch}
\nomenclature[S,38]{$\lambda_s$}{The slack stretch, the stretch needed to straighten the collagen fiber crimp}
\nomenclature[S,39]{$\mathcal{F}$}{Objective function for parameter estimation}
\nomenclature[S,40]{$\mathbfcal{I}$}{The information matrix}
\nomenclature[S,41]{$\mathbf{J}$}{The Jacobian of the objective function}
\nomenclature[S,42]{$\mathbfcal{H}$}{The Hessian of the objective function}
\nomenclature[S,43]{$\mathbf{\xi}$}{Vector of material parameters}



%-----------------------------------------------------------
%	Key Terms
%-----------------------------------------------------------
\nomenclature[K,1]{\it Phenomenological model}{Constitutive models that reproduces the mechanical response of materials without taking into considerations any underlying structure or mechanisms}
\nomenclature[K,2]{\it Micro-models}{Constitutive models utilizing structures and mechanism at a lower scale to predict the mechanical response at a higher scale}
\nomenclature[K,3]{\it Structural model}{Constitutive models that utilize the meso-scale microstructures to predict the mechanical response of soft tissues}
\nomenclature[K,4]{\it Effective constitutive model}{A computationally phenomenological model used to reproduce the response of micro material models in numerical simulations}
\nomenclature[K,5]{\it Polynomial series family}{Effective model forms composed primarily of sums of polynomial functions}
\nomenclature[K,6]{\it Separated exponential family}{Effective model forms composed primarily of sums of exponential functions}
\nomenclature[K,7]{\it Single exponential family}{Effective model forms composed primarily of a single exponential function of sums of polynomial functions}
\nomenclature[K,8]{\it ODF}{Orientation distribution function}
\nomenclature[K,9]{\it RDF}{Recruitment distribution function, the distribution of stretched needed to straighten the collagen fibers}
\nomenclature[K,10]{\it Physiologic range}{The range of deformations most likely to contain the physiologic loading path}





\printnomenclature





