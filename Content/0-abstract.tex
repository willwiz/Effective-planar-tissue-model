%%%%%%%%%%%%%%%%%%%%%%%%%%%%%%%%%%%%%%%%%%%%%%%%%%%%%%%%%%%%%
%%  Abstract												%
%%%%%%%%%%%%%%%%%%%%%%%%%%%%%%%%%%%%%%%%%%%%%%%%%%%%%%%%%%%%%
	One of the most crucial aspects of organ and system biomechanical simulations that seek to predict the outcomes of disease, injury, and surgical interventions is the underlying tissue constitutive model. Current soft tissue constitutive modeling approaches have become increasingly complex, often utilizing meso- and multi-scale methods for greater predictive capability and linking to the underlying mechanisms. However, such modeling approaches are associated with substantial computational costs. One solution to this problem is to use effective constitutive models, which reproduces the essential responses, but not the underlying mechanisms. Effective constitutive models can be implemented in place of meso- and multi-scale models in numerical simulations, but derive their responses from homogenizing the responses of the underlying meso- or multi-scale models. An robust effective constitutive model can thus drastically speed up simulations for a wide range of meso- and multi-scale models. However, there is no general consensus on how to develop and implement a single effective constitutive model form for a wide range of soft tissue responses. In the present study, we developed a robust effective constitutive model form, which can fully reproduce the response of a wide range of planar soft tissue responses, along with methods for fast-convergent parameter estimation. We evaluated this approach and demonstrated that it is able to handle materials of widely varying degrees of anisotropy, such as exogenously cross-linked bovine pericardium and aortic valve leaflet. Furthermore, we demonstrated the performance of this effective constitutive model in facilitating the finite element simulation of heart valves. This effective constitutive model approach has shown significant potential to improving the computational efficiency and numerical robustness of multi-scale and meso-scale modeling approaches, facilitate the application of inverse modeling, and simulations of growth and remodeling of soft tissues and organs.





